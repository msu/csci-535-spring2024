\documentclass{article}
\usepackage{../fasy-hw}

%% UPDATE these variables:
\renewcommand{\hwnum}{3}
\title{Computational Topology, Homework \hwnum}
\author{\todo{your name here}}
\collab{\todo{list your collaborators here}}
\date{due: 28 April 2022}

\begin{document}

\maketitle

This homework assignment should be submitted as a single PDF file to D2L.

General homework expectations:
\begin{itemize}
    \item Homework should be typeset using LaTex.
    \item Answers should be in complete sentences, and make sense without
        seeing the question.
    \item You will not plagiarize, nor will you share your written solutions
        with classmates.  (But, discussing the questions is highly encouraged).
    \item List collaborators at the start of each question using the
        \texttt{collab} command.
    \item Put your answers where the \texttt{todo} command currently is (and
        remove the \texttt{todo}, but not the word \texttt{Answer}).
    \item If you are asked to come up with an algorithm, you are
        expected to give an algorithm that beats the brute force (and, if possible, of
        optimal time complexity). With your algorithm, please provide the following:
        \begin{itemize}
            \item \emph{What}: A prose explanation of the problem and the algorithm,
                including a description of the input/output.
            \item \emph{How}: Describe how the algorithm works, including giving
                psuedocode for it.  Be sure to reference the pseudocode
                from within the prose explanation.
            \item \emph{How Fast}: Runtime, along with justification.  (Or, in the
                extreme, a proof of termination).
            \item \emph{Why}: Justify why this algorithm works.  At a minimum, I
                expect a statement of the loop invariant for each loop, or
                recursion invariant for each recursive function.
        \end{itemize}
\end{itemize}


\nextprob{Mock Review}
Choose one of the practice problems that either has you come up with an
algorithm or write a proof.  Write it up nicely, then exchange with a friend
(can be in class on Thursday (25 April), outside of class with a classmate, at
the writing workshop on Tuesday (23 April), or with someone outside of this
class). Briefly summarize their feedback and anything you learned in the
process.  Then, revise your proof accordingly.  Turn in the summary of their
feedback, along with the solution before and after feedback. You can repeat this
process a couple times with different people.

\nextprob{Biography}

Choose a researcher in computational topology and write a history of their
academic contributions.  Tell the story of how their research evolved and what
their major contributions are to the field of computational topology.  The
academic history should be written for a technical audience.

\nextprob{Research Dive}
The \href{https://topology.ima.umn.edu/seminars}{AATRN} has an archive of
several talks per semester since Fall 2014.  Choose one of the talks and
watch it, then answer the following questions:
\begin{enumerate}[(a)]
    \item Summarize the talk in about 1/2 page.
    \item Choose a published paper by the speaker; let's
        call this Paper-1.  First,
        skim the paper to get a general idea of the paper, then try to
        read as much detail as you can.  Describe where you got lost,
        using up to one-half page.
    \item Identify a paper that would help enhance your understanding of
        Paper-1; let's call this Paper-2.  This paper can perhaps be
        found as a reference in Paper-1, or from your mad googling
        skills. Skim the paper to get a general idea of the paper, then try to
        read as much detail as you can.  Describe where you got lost in
        Paper-2,
        using up to one full page.
    \item Repeat this process one more time: identify Paper-3 that would
        help you resolve where you got lost in Paper-2. Skim the paper,
        then read it in as much detail as you can: where do you get lost
        here?  Does this process have an end?
\end{enumerate}



\end{document}
