\documentclass{article}
\usepackage{../fasy-hw}

%% UPDATE these variables:
\renewcommand{\hwnum}{3}
\title{Computational Topology, Homework \hwnum}
\author{\todo{your name here}}
\collab{\todo{list your collaborators here}}
\date{due: 18 April 2024}

\begin{document}

\maketitle

\input{../directions}

\nextprob{Function Space}
% \collab{if applicable, update collab list}
Let $(X,d)$ be a metric space.  That is, $X$ is a set and $d \colon X \times X
\to \R$ is a distance metric on $X$.  A \emph{metric ball} at $x \in X$ with
radius $r \geq 0$ is the (open) set: $B_r(x) := \{x' \in X ~|~ d(x,x') < r \}$.
We can use the set of all metric balls to generate a
topology on $X$.  When we do so, we call this a \emph{metric} topology on $X$.

One of my favorite types of topological spaces are where the points represent
functions.  For example, let~$(X,\mathcal{T}_X)$ be a topological space
and let $(Y,\mathcal{T}_Y)$ be
a metric space (corresponding to the distance function~$d_Y
\colon Y \times Y \to \R$).  Let $C(X,Y)$ denote the set of all continuous
functions from~$X$ to~$Y$.  We can topologize~$C(X,Y)$ using the $L_{\infty}$-metric; that is,
we define a distance metric $\ell_{\infty} \colon C(X,Y) \times C(X,Y) \to
\R$~by
$$\ell_{\infty}(f,g) := \sup_{x \in X} d_Y(f(x),g(x)).$$

\begin{enumerate}[(a)]

    \item Prove that this is a metric.

        \paragraph{Answer}
        \todo{answer here}

    \item Suppose the topology on $X$ is the indiscrete topology; that is,
        $\mathcal{T}_X=\{\emptyset,X\}$.  Describe the topological space whose
        set is $C(X,\R)$ and whose topology is generated by metric balls in
        $\ell_{\infty}$. (Note: when I use $\R$ without explicitly stating the
        topology, please assume that we are using the standard topology).

        \paragraph{Answer}
        \todo{answer here}

\end{enumerate}

\nextprob{Two-Coloring}
% \collab{TODO - uncomment if your collaborators on this have changed
Answer HE-CT Part II, Question~$2$~(\emph{$2$-Coloring}).

\paragraph{Part (i) Answer}
\todo{answer here}

\paragraph{Part (ii) Answer}
\todo{answer here}

\nextprob{Path Spaces}
% \collab{TODO - uncomment if your collaborators on this have changed

Consider the (total) path space from $(0,0,\ldots,0)$ to $(3,3,\ldots,3)$ in $\R^d$,
where there is exactly one obstacle: the central $d$-cube is missing. In $\R^2$,
we saw this path space is homotopy equivalent to two points. In $\R^3$, we saw
that the path space is homotopy equivalent to $\S^1$.  Conjecture what the path
space in $\R^d$ is, providing some justification (not necessarily a full proof)
for your claim.

\paragraph{Answer}
\todo{answer here, hopefully with some figures}

\nextprob{Nerve Lemma}

The Nerve Lemma actually has many different forms.  Give three variants of the
Nerve Lemma.  (Note: you are expected to do some digging. If you do not restate
the theorem in your own words, be sure to make it clear that the lemma is taken
as stated elsewhere. However, the theorem should be written in a way that should
be understandable to a classmate in this class).

\paragraph{Answer}
\todo{answer here}

\end{document}
